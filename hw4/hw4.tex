\documentclass{article}

\usepackage[margin=1in]{geometry}
\usepackage{fancyhdr}
\usepackage{amsmath}
\usepackage{amssymb}
\usepackage{listings}
\usepackage{graphicx}
\usepackage{enumerate}
\usepackage{tikz}
\usetikzlibrary{automata,positioning}


\pagestyle{fancy}

\lhead{CS 373 \\ Homework 4}
\chead{}
\rhead{Drew Cross (ddcross2) \\
    \emph{Partners:} Eric Parsons, Bryan Plumer}
    \lfoot{}
    \cfoot{}
    \rfoot{}

    \begin{document}

    \subsection*{Problem 1}
    CFG Parsing

    RNA molecules are crucial to all life on Earth.  In fact, they're so important, some
    researchers theorize these molecules may have been the start of all life on Earth. For our
    purposes, RNAs are represented as strings on the alphabet $\Sigma=\{a,u,c,g\}$.  The following 
    grammar comes from an actual research pape
    \footnote{http://www.biomedcentral.com/1471-2105/5/71} about computational analysis of RNA: \\

    $S \rightarrow LS \;|\; L$

    $L \rightarrow aFu \;|\; uFa \;|\; cFg \;|\; gFc \;|\; a \;|\; u \;|\; c \;|\; g$

    $F \rightarrow aFu \;|\; uFa \;|\; cFg \;|\; gFc \;|\; LS$


    \begin{enumerate}[(a)]
\item Put this grammar in CNF. (3 Points)

    Notice this grammar is $\Sigma^+$ by using rule $S$ repeatedly, then immediately
    going to terminal rules in $L$, thus can be generated by the following grammar (in CNF):

    $S_0 \rightarrow S$

    $S \rightarrow LS \;|\; a \;|\; u \;|\; c \;|\; g$

    $L \rightarrow  a \;|\; u \;|\; c \;|\; g$

    \item Provide the CYK table for these strings:

    \begin{itemize}
\item $w_1=gaccuc$ (4 points)

    %TODO
    \begin{tabular}{|c|c|c|c|c|c|c|}
    \hline
    LS&-&-&-&-&-\\
    \hline
    LS&LS&-&-&-&-\\
    \hline
    LS&LS&LS&-&-&-\\
    \hline
    LS&LS&LS&LS&-&-\\
    \hline
    LS&LS&LS&LS&LS&-\\
    \hline
    LS&LS&LS&LS&LS&LS\\
    \hline
    g&a&c&c&u&c\\
    \hline
    \end{tabular}
\item $w_2=uggu$ (3 points)

    %TODO
    \begin{tabular}{|c|c|c|c|}
    \hline
    LS&-&-&-\\
    \hline
    LS&LS&-&-\\
    \hline
    LS&LS&LS&-\\
    \hline
    LS&LS&LS&LS\\
    \hline
    u&g&g&u\\
    \hline
    \end{tabular}
    \end{itemize}

\item Prove that the language accepted by this grammar is actually regular. (3 Points)

    This language is regular because it is $\Sigma^+$.

    \item Can you explain why these researchers are interested in parsing a regular language with
a CFG? (2 points)

    Because the parsing of this grammar yields the structure for an RNA molecule.
    \end{enumerate}

    \newpage
    \subsection*{Problem 2}
    Provide two proofs that the following languages are not context-free. First, provide a proof
    using the the pumping lemma for CFLs. Second, provide a proof using closure properties of
    CFLs. You may assume that the languages Sipser used when demonstrating the pumping lemma
    (see Sipser Examples 2.36, 2.37, and 2.38 pp. 126) are known to be non-context-free.

    \begin{enumerate}[(a)]
    \item $A = \{ 0^n 1^{2m} 0^{m+k} \;|\; 0 \le n \le m, k \ge 0 \}$

    Assume A is a context free language.
    Let $p$ be the pumping length, select $s = 0^p1^{2p}0^p$ where $s = uv^ixy^iz$ is in the
    language.

    Due to $|vxy| \leq p$ we must have one of the following cases:

    \begin{enumerate}[{Case} 1:]

    \item

    If $y$ or both $v$ and $y$ are in the $0^p1^2p$ then when pumping up the string
    it increases the number of 0's, 1's, or both in the beginning. In this case there will be
    too few 0's at the end.

    \item

    If $y$ or both $v$ and $y$ are in the $0^p1^2p$ then when pumping down the string
    it decrease the number of 0's, 1's, or both in the end. In this case there will be
    too few 0's at the beginning.

    \item

    If $vxy$ contains at least one 1 and one 0 from the trailing zeros, when
    pumping down we will decrease either the 1's, 0's, or both. Since the leading
    zeros do not get changed, the pumped string is not in $A$ 

    Since we cannot pump this string we have a contradiction and the language
    must not be context free.

    \end{enumerate}


%    $00 ... 00 \; 11 ... 11 \; 00 ... 00$

    \[ \square \]

    Closure:

    $0^* \circ \{ 0^n 1^{2m} 0^{m+k} \;|\; 0 \le n \le m, k \ge 0 \}$

    This concatenation breaks the rule where $n \leq m$ because we can add an arbitrary number
    of zeros to the beginning of the string now we don't need
    to restrict $n \leq m$ therefore we have:

    $\{0^i1^m0^k | i,j,k \geq 0\}$ Note: k here is not the same as in the previous step.

    Define a homomorphism $h(0) = (a \cup c)$ and $h(1) = b$ now we have:

    $\{(a \cup c)^ib^m(a \cup c)^k | i,j,k \geq 0\} \cap L(a^*b^*c^*)$

    This leaves us with:

    $\{a^ib^mc^k | i,j,k \geq 0\}$

    Which is not a CFL(Sipster 2.37)

    \[ \square \]

    \item $B = \{w\in \{a,b,c\}^* \;|\; \#a(w)=\#b(w)=\#c(w)\}$

    Assume B is a CFL. Let $p$ be the pumping length, and select $s = a^pb^pc^p$

    Due to the condition  $|vxy| \leq p$, $vxy$ must be somewhere in the following:

    $a^p, b^p, c^p, a^pb^p$ or $b^pc^p$

    By pumping up in any of these cases we are left with strings that have an
    unequal number of a's, b's or c's. Therefore we have a contradiction,
    and the language is not context free.
    \[ \square \]

    Closure:

    $\{w\in \{a,b,c\}^* \;|\; \#a(w)=\#b(w)=\#c(w)\} \cap L(a^*b^*c^*)$

    Only allows a number of a's followed by the same number of b's ten the same
    number of c's after that, which is:

    $\{a^nb^nc^n | n \geq 0\}$

    Which is not a CFL(Sipster 2.36)
    \[ \square \]

    \end{enumerate}

    \newpage
    \subsection*{Problem 3}
    In class, we’ve covered several classes in the Chomsky Hierarchy: regular, context-free, and
    context-sensitive languages. Based on what you’ve learned, it should be possible to assign
    languages from these classes to their most exclusive (or least general) class in the Chomsky
    Hierarchy. For example, we know the language L = f0 n 1n 0g is context-free, but not regular.
    We used the pumping lemma for regular languages to prove that it is not regular (see Sipser,
            Example 1.73, pp. 80) and wrote a CFG to prove that it is context-free (see Sipser, Section
                2.1, pp. 100). Using all your knowledge of formal languages, place these languages into the
            most exclusive class in the hierarchy you know. State the class, and provide a proof that it is
            a member of that class. Also provide a proof that it is not a member of a more exclusive
            class. None of these languages belong exclusively to classes we haven’t yet covered, and all
            of the languages are on the alphabet $\Sigma = \{0,1\}$.
            \begin{enumerate}[(a)]
            \item $L = \{w \;|\; \#0(w)=\#1(w)\}$ (5 points)

            L is a CFL:

            Not Regular:

            $L \cap {0^*1^*} = 0^n1^n$
            \[ \square \]

            Is a CFL:

            $S \rightarrow 0S1 \;|\; 1S0 \;|\; \varepsilon$
            \[ \square \]

            \item $L = \{w \;|\; w=w^r, \#0(w)=\#1(w)\}$ (5 points)

(Proof borrowed from fa08 solutions):

Suppose $L$ is context free. Let $p$ be the pumping length. Consider the string
$s = 0^p1^p1^p0^p$.

$s$ can be split into $uvxyz$ such that $|vxy| \leq p$ and $|xy| \geq 0$, and
$uv^ixy^iz \in L$ for all $i \geq 0$. Since $|vwx| \leq p$ $vwx$ will consist
entirely of 0's, entirely of 1's or is of the form $0^m1^n$ or $1^n0^m$

This only allows for the following cases:

\begin{enumerate}[{Case} 1:]

    \item $vxy$ lies only in one of the first 0's, the first 1's, the second 1's
    or the second 0's (doesn't straddle a boundary). In these cases pumping down
    on the string results in a string that is no longer a palindrome. For each
    of the respective cases, you're removing 0's or 1's on only one size of the
    string, making it no longer a palindrome.

    \item $vxy$ straddles a boundary, that is, $vxy$ lies in either $0^p1^p$ or
    it lies in $1^p0^p$. In these two cases, when you pump down, you remove sets
    of zeros and ones from one side, that still exist in the other.

\end{enumerate}

Since we cannot pump this string, we have a contradiction, so $L$ must not be
context free.
\[ \square \]

Proof by construction:
We know that the language that generates all even length palindromes is a CSL.

$S \rightarrow 1A\# | 0A\$$

$A \rightarrow 1A\# | 0A\$ | 0\$ | 1\#$

$1\# \rightarrow 11$

$1\$ \rightarrow 10$

$0\# \rightarrow 01$

$0\$ \rightarrow 00$

We also know that the language that generates all strings with the same number
of 1's and 0's is a CSL.

$S \rightarrow 0A1A | 1A0A$

$A \rightarrow 0A1A | 1A0A | 0C1C | 1C0C$

$C \rightarrow 01 | 10$

If we take the intersection of these two languages, we get a CSL that is
the language of all palindromes that have the same number of 1's and 0's.

\[ \square \]

            \item $L = \{w=xy \;|\; x,y\in\Sigma^* \text{ and } \#0(x)+\#1(x)=\#0(y)+\#1(y)\}$ (5 points)

            L is Regular:

            $(\Sigma\Sigma)^*$
            \[ \square \]

            \end{enumerate}
            \newpage
            \subsection*{Problem 4}
            As on the midterm, the shuffle operation combines two strings $x$ and $y$ in all possible
            
            ways that preserve the original order of symbols from $x$ and $y$.  For example, if
            $x=x_1x_2x_3$ and $y=y_1y_2y_3y_4$, some valid shufflings are $y_1x_1y_2x_2x_3y_3y_4$ and
            $x_1x_2y_1y_2y_3x_3y_4$.  We define the shuffle of two languages $shuffle(A,B)$ to be the
            shuffling of all the words in $A$ with all the words in $B$.  More explicitly,
            \[shuffle(A,B)=\bigcup\limits_{x \in A}\bigcup\limits_{y \in B} shuffle(x,y)\]
            \begin{enumerate}[(a)]
    \item Prove that, if $A$ is context-free and $B$ is regular, $shuffle(A,B)$ is
context-free. (8 points)
    %TODO
    \item Prove that, if $A$ and $B$ are context-free, $shuffle(A,B)$ is not necessarily
context-free. (8 points)
    %TODO
    \item For a binary language $L$ over $\Sigma=\{0,1\}$, we define the function $p$ (which
            maps a single string to a set of strings) as follows:
    \[p(w) = \{ x \in \{0,1\}^* \;|\; \#0(w) = \#0(x), \#1(w) = \#1(x) \}\]
    %TODO
    Using $p$, we define the operation $p$ on languages:

    \[p(L) = \bigcup\limits_{w \in L} p(w)\]

Let $L$ be a regular language.  Show that $p(L)$ is context-free. (9 points)
    \end{enumerate}
    \newpage
    \subsection*{Problem 5}

    Give a context sensitive grammar for the language:
    \[A = \{0^k \;|\; k = 2^i+1,i\ge0\}\]

Using a noncontracting grammar:

    $S \rightarrow C\$ | 00$

    $C \rightarrow AC | A$

    $A \rightarrow B\#$

    $AB \rightarrow BBA$

    $AB\# \rightarrow BB\#$

    $B\# \rightarrow D0$

    $BD0 \rightarrow 000$


    \newpage
    \subsection*{Problem 6}
    Give a "higher level" (see "Examples of Turing machines", Sipser pp. 142) description of a
    Turing machine for the language:

    \[A=\{a^ib^jc^k | i\%j=k \text{ and } i,j\ge1, k \geq 0\}\]

    Where $\%$ is the modulo operator.

    \begin{enumerate}[1)]

    \item If first symbol is blank, reject, otherwise mark it with a hat.

    \item Scan right until the first blank symbol. If the string was not of the form
    $a^*b^*c^*$ and there is not at least one $a$ and one $b$ reject.

    \item Scan left to the first $b$. Shuttle back and forth between $b$'s and 
    $c$'s marking one $b$ and $c$ for each pair. If there are not enough $b$'s
    to mark all $c$'s, reject.

    \item Now scan right until a blank, then scan left until returning to the
    hatted symbol, unmaking $b$'s and $c$'s along the way. Shuttle between $a$'s
    and $c$'s marking each pair. If you run out of $a$'s before $c$'s reject.

    \item Scan left to the hatted symbol then scan right to find the first
    unmarked a. If no unmarked a exist, accept.

    \item Shuttle left and right to mark pairs of $a$'s and $b$'s if there
    are not enough $a$'s, reject.

    \item Scan right to first blank, then left unmarking all $b$'s. Repeat from
    step 5.

    \end{enumerate}

    \end{document}
