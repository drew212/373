\ifx\bookMode\undefined
\documentclass[12pt]{article}
\usepackage{373,graphicx}
\fi

\newcommand{\figurescale}{0.8}

\BeginDocument

\HWNum{CS373 - Spring 2012}{0}{Thursday Feb 8 at 2:00 PM in class (151 Everitt Lab)}
\Title{Problem Set 1}
\MakeTitleHW

%\hline

Please \underline{follow} the homework format guidelines posted on the class web
page:

\centerline{\url{\CourseWebpage}}
%\hline
\bigskip

\begin{enumerate}

%Problem 1
    \item \ProblemTtl{Encoding input and building DFAs}{Design}{15}
    
    Design a DFA that accepts each of the following languages on $\Sigma=\{0,1\}.$
    \begin{enumerate}[a)]
    	%\item $\{w \;|\; w \text{ starts with the string 01 concatenated to itself one or more times, }$\\$\text{followed by zero or more 1's}\}$ (3 Points)
	\item $\{w \;|\; w \text{ does not contain three consecutive 1's and does not start with 00} \}$ (5 Points)
	\item $\{w \;|\; \text{the total number of 0's in }w \text{is congruent to 2 modulo 3 and the total }$\\$\text{number of 1's in }w\text{ is congruent to 4 modulo 5}\}$ (5 points)
	\item $\Sigma^{*}\setminus((\Sigma\Sigma\Sigma)^{*}\cup(\Sigma\Sigma)^{*}\cup\emptyset^{*})$ (5 Points)
    \end{enumerate}

%Problem 2
    \item \ProblemTtl{NFA to DFA conversion}{Comprehension}{10}
    
    Convert the following NFAs to equivalent DFAs without any unreachable states (5 points each).
    \begin{enumerate}[a)]
	\item \hspace*{\fill} \newline \begin{center}\includegraphics[scale=1]{2-1.pdf}\end{center}
	\item \hspace*{\fill} \newline \begin{center}\includegraphics[scale=1]{2-2.pdf}\end{center}
    \end{enumerate}
    
%Problem 3
    \item \ProblemTtl{Closure on regular languages}{Proof}{15}
    
    Prove that regular languages are closed under the following operations:
    \begin{enumerate}[a)]
	\item $f(L)=\{x:xy \in L \text{ for some } y\in\Sigma^*\}$ (5 Points)
	%\item $g(L)=\{x:xy \in L \text{ for some } y\in\Sigma^*\}$
	\item $r(L)=\{x^R:x\in L\}$ (5 points)
	\item $Q(L_1,L_2)=\{w:wx \in L_1 \text{ and } x\in L_2\}$ (5 points)
    \end{enumerate}
    
%Problem 4
    \item \ProblemTtl{FSTs}{Design}{20}
    
    A \emph{finite state transducer} (FST) is a modified DFA that, rather than outputting \emph{accept} or \emph{reject}, produces an output string.  You can think of it as a translating automaton that converts an input string into an output string.  The transitions of an FST are labeled with an \emph{input} symbol and an \emph{output} symbol, which are separated by a $/$.  There are no accept states.  Here is an example FST:
    \newline
    \begin{center}\includegraphics[scale=1]{3.pdf}\end{center}
    For the input string $1010$, this example FST outputs $aaaa$.  For input string $0000$, it outputs $bbbb$.  A formal description of an FST must have:
    \begin{itemize}
    	\item $Q$ a finite set of \emph{states},
	\item $\Sigma$ a finite \emph{input alphabet},
	\item $\Gamma$ a finite \emph{output alphabet},
	\item $d: Q\times\Sigma\rightarrow Q\times\Gamma$ a \emph{transition function}, and
	\item $q_0 \in Q$ a \emph{start state} 
    \end{itemize}
    FSTs are therefore a 5-tuple: $F=(Q,\Sigma,\Gamma,d,q_0)$
    \begin{enumerate}[a)]
	\item Let 
	\begin{itemize}
		\item $\Sigma=\{ \begin{bmatrix} 0 \\ 0 \end{bmatrix}, \begin{bmatrix} 0 \\ 1 \end{bmatrix}, \begin{bmatrix} 1 \\ 0 \end{bmatrix}, \begin{bmatrix} 1 \\ 1 \end{bmatrix} \} $
		\item $\Gamma=\{0,1\}$
	\end{itemize}
	Give a formal description of an FST that treats an input string as two binary numbers (one on top, the other on the bottom) written in reverse order and outputs their sum in reverse binary, ignoring overflows (i.e. the input and output should always be the same length.)  For example, $\begin{bmatrix} 0 \\ 0 \end{bmatrix}\begin{bmatrix} 1 \\ 0 \end{bmatrix}\begin{bmatrix} 1 \\ 1 \end{bmatrix}$, the top number is $6$ in decimal and the bottom number is $4$ in decimal.  The FST you design should output $010$ (decimal $10$, after accounting for the overflow and putting it in reverse binary.) (5 points)
	\item Let $\Sigma_1=\{a,b\}$ and $Q_n=\{q_1,q_2,...,q_n\}$ be sets.
	
	Let $\mathbb{D}_n=\{D_n\;|\;D_n=(Q_n,\Sigma_1,d:Q_n\times\Sigma_1\rightarrow Q_n,q_0\in Q_n,F\subseteq Q_n)\}$ be the set of all DFAs with $n$ states.
	
	\begin{enumerate}[i)]
		\item Create an encoding for $\mathbb{D}_n$ using the symbols from $\Sigma_2=Q_n \cup \{\$\}$ (here, $\$$ is just one additional symbol you can use.)  In other words, create a one-to-one function $M: \mathbb{D}_n\rightarrow\Sigma_2^{*}$. (3 points)
		\item Give a formal description of an FST that takes any encoded $D_n\in\mathbb{D}_n$, concatenated with a string $w\in\Sigma_1^{*}$ and outputs the states visited by $D_n$ when processing $w$.  If the input string does not follow this format, your FST should output the empty string.  The transitions in your FST may have $\varepsilon$ on the input or output. (12 points)
	
	\end{enumerate}
    \end{enumerate}

%Problem 5
    \item \ProblemTtl{DFAs for a language}{Proof}{20} 
    
    Consider some regular language $A$ over an alphabet $\Sigma$. Prove that the set of all DFAs accepting $A$ that use input alphabet $\Sigma$ is countably infinite.
    
%Problem 6
    \item \ProblemTtl{Kleene star}{Proof}{20}
    \begin{enumerate}[a)]
    	\item Prove that $L^*L^*=L^*$ for any language L. (5 Points)
	\item Prove that $L^{**}=L^*$ for any language L. (5 Points)
    	\item Using only a description of a DFA, prove that the Kleene star of any finite language $L$ is regular. (10 Points)
   \end{enumerate}
\ end{enumerate}
\EndDocument
