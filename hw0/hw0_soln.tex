\documentclass{article}

\usepackage[margin=1in]{geometry}
\usepackage{fancyhdr}
\usepackage{amsmath}
\usepackage{amssymb}
\usepackage{listings}
\usepackage{graphicx}
\usepackage{enumerate}


\pagestyle{fancy}

\lhead{CS 373 \\ Homework 0}
\chead{}
\rhead{Drew Cross (ddcross2)\\ \emph{Partners:} Sean Nicolay, Eric Parsons }
\lfoot{}
\cfoot{}
\rfoot{}

\begin{document}

\section*{Homework 0}

\subsection*{Problem 1}

\begin{enumerate}[1)]

            \item $\mathbb{P} \in P(\mathbb{N}) \rightarrow$ True
        \item $\mathbb{Z}\cup \mathbb{N} \subsetneq \mathbb{Z} \rightarrow$ False
            \item $(\mathbb{Z} \setminus \mathbb{P}) \cup \mathbb{N} = 
                (\mathbb{Z} \setminus \mathbb{N}) \cup \mathbb{P}\rightarrow$ False

      \item $(\mathbb{N}\cap \{-1\}) \in \{\varnothing\}\rightarrow$ True
            \item $\varnothing \in \{\{\varnothing\}, \mathbb{Z}\}\rightarrow$ False
            \item $\varnothing \subseteq \{\{\varnothing\}, \mathbb{Z}\}\rightarrow$ True
            \item $|P(P(\varnothing)) \setminus \{\varnothing\}|=1\rightarrow$ True
        \item $\varnothing = \mathbb{N}^2 \setminus ((\mathbb{N} \cap \mathbb{Z} )
        \times (\mathbb{N} \cup \mathbb{P}))\rightarrow$ True

            \item $P(\mathbb{Z}) \times P(\mathbb{N}) \neq P(\mathbb{Z} \times \mathbb{N})\rightarrow$ True
            \item $\{w\mid w=w_{0}w_{1}...w_{n},w_{i} \in \mathbb{Z}\}$ is a language. $\rightarrow$ False
\end{enumerate}

\pagebreak
\subsection*{Problem 2}

    The following is a proof that all natural numbers are equal.

    Let $a,b\in\mathbb{N}$ and $k = \max(a,b)$. We will prove, using induction on $k$, that $a = b$.

    Base case: $k = 1$.  In this case, since $a$ and $b$ are natural numbers and $a,b\le 1$,

    both of them are equal to 1

    Inductive hypothesis:  If $a,b\in\mathbb{N}$ and $k = \max(a,b) \le n$, then $a = b$.

    Inductive step: Let $k = n+1$. Then $\max(a-1,b-1) = \max(a,b)-1 = k-1 = n$.

    By the inductive hypothesis, $a-1 = b-1$; therefore $a = b$.

    Conclusion:  Since $a$ and $b$ were arbitrary, we have shown that any 2 natural numbers are equal.

    Explain in detail why this proof is wrong.\\[5pt]

    The main issue with this proof is the inductive step. In the inductive step, when $k=n+1$ there are three
    posibilities for what can happen to max(a,b):
    \begin{enumerate}[1)]
    \item a, b both increase $\rightarrow \max(a-1,b-1) = k-1 = n$ - The case considered in this proof
    \item a only increases $\rightarrow \max(a-1,b) = k-1 = n$
    \item b only increases $\rightarrow \max(a, b-1) = k-1 =n$
    \end{enumerate}

    Since this proof doesn't consider all 3 cases, it is invalid.

\pagebreak
\subsection*{Problem 3}

Let $\mathbb{R}$ denote the set of real numbers. Prove that a set of disjoint open intervals of $\mathbb{R}$
is at most countably infinite.\\

In other words, let $X \subset P(\mathbb{R})$ with elements of the form:
\[ (a,b)=\{ x\in\mathbb{R} \mid a<x<b \} \text{ where } a,b \in \mathbb{R}, a<b.\]

Let $\mathbb{Q}$ denote the rational numbers. Because the real numbers lie on an interval between $a$ and $b$, it is
possible to pick a number $q\in\mathbb{Q}$ such that $q > a$ and $q < b$. To show this is possible we can find a decimal
location where the two real numbers differ, call it $n$. We can then truncate the real $b$ at $n+1$.
Now we have a number with a finite number of decimal places, so we have found a $q_{1} \in \mathbb{Q}$.
Now let $q = q_{1}$. Since we have decided to select a decimal location
we know that $a > q_{1} \leq b$. (The fact that $q_{1}$ can possibly be equal to $b$
and does not lie on the interval is irrelevant because it still uniquely defines the interval.
In other words, no two intervals can share the same b, because they are disjoint.) Because we have selected $q = q_{1}$
we know that $q$
lies on the interval $(a, b]$, therefore we have a one-to-one mapping from $\mathbb{Q}$ to $X$.\\


We proved in class that $\mathbb{Q}$ is countable, and since we have found a mapping from $\mathbb{Q}$ to $X$
we have proved that $X$ is at most countably infinite.
\[\square\]

\pagebreak
\subsection*{Problem 4}
We have a completely connected graph with edges labled either $\alpha$ or $\beta$. Using induction we can prove that
we can remove one type of edge, either $\alpha$ or $\beta$. And still have a connected graph.\\

Base Case:\\
A graph containing a single node:\\
This case is trivial because we can remove either edge type $\alpha$ or $\beta$.\\

Inductive Hypothesis:\\
Given a graph of $n$ nodes, assume we can pick an edge type, either $\alpha$ or $\beta$ to remove, and still have a
connected graph.\\

Inductive Step:\\
If we add an addional node: node $n+1$, we need to prove we can still pick one edge type to remove and still have a
connected graph. We now have two cases:\\

Case 1:\\
All paths from the node $n+1$ are of one type, either $\alpha$ or $\beta$. In this case, we can remove the opposite
edge type. Now, since all other nodes are connected to node $n+1$, the graph is still connected.\\

Case 2:\\
The paths from node $n+1$ contain at least one of each $\alpha$ and $\beta$. By our inductive hypothesis we know that
we're able to remove at least one edge type from the graph. Without the loss of generality we can assume this type is
$\alpha$. In this case, we know that the node $n+1$ is connected by at least 1 $\beta$ edge, so we can remove the
$\alpha$ edge type and still have a connected graph.
\[\square\]

\end{document}
